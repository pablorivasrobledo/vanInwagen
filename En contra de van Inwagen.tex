\documentclass[12pt]{article} % Font size (can be 10pt, 11pt or 12pt) and paper size (remove a4paper for US letter paper)

\usepackage[a4paper, total={6in, 8in}]{geometry}
\setlength{\parskip}{0.5em}
\setlength{\parindent}{1.5 em}
\renewcommand{\baselinestretch}{1.5}
\setlength{\skip\footins}{1.2pc plus 5pt minus 2pt}
\usepackage{blindtext}
\usepackage[hang]{footmisc}
\usepackage{csquotes}
\usepackage{graphicx} % Required for including pictures
\usepackage{wrapfig} % Allows in-line images

\usepackage{enumitem} % Allows to resume lists

\usepackage{fontspec}
\usepackage{amssymb}
\usepackage{amsmath}

\usepackage{hyperref}
\hypersetup{
	colorlinks=true,
	linkcolor=red,
	filecolor=magenta,      
	urlcolor=cyan,
}
\usepackage{lplfitch}

\usepackage{polyglossia}%Always use with \setmainlanguage
\setmainlanguage {spanish}%changes the document interface language to spanish


\usepackage[backend=biber, style=apa]{biblatex}
\addbibresource{bib-vanInwagen.bib}

\title{Tres argumentos en contra del incompatibilismo de van Inwagen}
\author{Martín Buenahora Bonilla\\
		Juliana Ocampo Guzmán\\
		Pablo Rivas Robledo \\
		$\prec\between\succ$\\
       	Universidad de La Sabana\\}
         
\date{}


\begin{document}
\maketitle	


\section{Introducción}
Peter van Inwagen es un filósofo estadounidense nacido en 1942. Su trabajo filosófico está enfocado en temas de metafísica, filosofía de la acción y filosofía de la religión y en su libro de 1983 \textit{An Essay on Free Will} contribuyó a reavivar el debate acerca de la compatibilidad entre el determinismo y el libre albedrío en el siglo XX \parencites[23]{kane-free_will}[sec.1]{sep-incompatibilism-arguments}. En este libro, así como en otros trabajos, van Inwagen defiende que los conceptos de determinismo y libre albedrío son incompatibles, ambos no pueden ser ciertos al tiempo. 

En este escrito presentaremos tres argumentos en contra del incompatibilismo de van Inwagen, en concreto, demostraremos que el sistema filosófico de van Inwagen no es lo suficiente poderoso para demostrar que haya incompatibilidad entre el determinismo y el libre albedrío: por un lado, el que no haya leyes naturales acerca de nuestro comportamiento mental XXXX; por otro lado, la visión de van Inwagen de un mundo determinismo no es suficiente para precluir una versión minimalista del libre albedrío: por último, la forma de van Inwagen formula la necesidad del pasado XXXX

Para tal fin, primer expondremos qué es el problema del compatibilismo y cómo lo enfrenta van Inwagen; luego expondremos nuestros argunentos en contra del mismo; después de eso pasamos a afirmar que rechazar el incompatibilismo de van Inwagen no nos hace compatibilistas y por ende no nos hace falta un defensa del compatibilismo; por último, exponemos algunas conclusiones.

\section{El problema del compatibilismo}
	\subsection{El incompatibilismo de van Inwagen}
	
\section{Tres argumentos en contra del incompatibilismo de van Inwagen}
	\subsection{Leyes de la naturaleza y leyes psicológicas: en contra del primer argumento}
	\subsection{Posibilidades no actualizadas: en contra del segundo argumento}
En esta sección pretendemos demostrar que el segundo argumento de van Inwagen no es suficiente para afirmar que el incompatibilismo sea cierto. El segundo argumento de van Inwagen o el Segundo Argumento Formal, como él mismo lo llama \parencite[83]{Inwagen1986}, es un argumento que demuestra la incompatibilidad del determinismo y el libre albedrío suponiendo el determinismo y demostrando que este no es compatible con la forma más débil de formular el libre albedrío, por lo que cualquier otra forma de formular el concepto sería incompatible con un mundo determinista.\footnote{A menos que se especifique lo contrario, el argumento y las definiciones que se expone en esta sección son tomados de \parencite[secs. 3.9-3.10]{Inwagen1986}}

El segundo argumento de van Inwagen requiere de lo siguientes elementos, los cuales primero enunciamos y luego explicamos con más detalle:

\begin{itemize}
	\item La noción de mundo posible;
	\item El mundo actualizado (\textit{the actual world});
	\item La propiedad diádica, simétrica, reflexiva y no transitiva de `compartir un tajada con’, la cual se da entre pares de mundos posibles: $Sxy$;
	\item La propiedad diádica, simétrica, reflexiva y transitiva de ‘ser nomológicamente congruente con’, la cual se da entre pares de mundos posibles: $Oxy$;
	\item La propiedad diádica, antisimétrica, antireflexiva e intransitiva de ‘tener acceso a’, la cual se da entre agentes y mundos posibles: $Hxy$;\footnote{Debe tenerse en cuenta que `$H$' no es la tradicional `$R$' usada en lógica modal alética para representar relaciones de accesibilidad entre mundos posibles, se trata de una relación introducida por van Inwagen que existe entre agentes y mundos posibles. Más adelante se trata con algo más de detalle.} y
	\item La propiedad monádica de mundos posibles de ser determinado (\textit{deterministic}): $Dx$.
\end{itemize}
Lo primero es la noción de mundo posible. Siguiendo a Plantinga \parencite*{Plantinga1974-PLATNO}, para van Inwagen los mundos posibles son ``members of the class of ways things  might be or possible ways things might be arranged or, simply, possibilities'' \parencite[80]{Inwagen1986}. Sin embargo, no cualquier posibilidad es un mundo posible, para que una posibilidad sea un mundo posible, debe ser una posibilidad comprensiva, es decir, que respecto de cualquier otra posibilidad la primera debe incluir o precluir la segunda: una posibilidad incluye a otra cuando es imposible que la primera se actualice mientras que la segunda no lo haga; una posibilidad precluye a otra cuando es imposible que ambas se actualicen. Toda posibilidad es un objeto abstracto, por lo que todo mundo posible es un objeto abstracto.

Continuamos con la noción de mundo actual. Para van Inwagen, esta es una descripción definida que designa a un objeto abstracto tal que es una posibilidad comprensiva actualizada . Es decir, es una posibilidad que incluye o precluye cualquier otra posibilidad y es el caso. Por otro lado, la propiedad `$S$' es una propiedad que comparten dos mundos posibles cualquiera que sean indistinguibles en un instante; así como la propiedad `$O$' es la equivalencia que existe entre dos mundos que tengan las mismas leyes de la naturaleza, dos mundos posibles no tienen esta propiedad si (1) no comparten exactamente las mismas leyes de la naturaleza (i.e. un mundo en donde no exista la ley de la gravedad no es nomológicamente congruente con el mundo actual) o (2) tienen las mismas leyes pero con distintos valores.

Por último, `$H$' expresa una relación atemporal entre agentes y mundos posibles, es decir, no necesita de un tercer elemento (instante del mundo posible) para que la relación se satisfaga si no por un par ordenado <agente, mundo posible>, así, `$Hxy$' se lee como ‘x tuvo, tiene o tendrá (en algún momento de su vida) acceso a y’. ‘D’, por otro lado, es una propiedad de un mundo posible si ese mundo posible es el único mundo posible con el cual comparte una tajada y es nomológicamente congruente.\footnote{Esta última definición puede resultar más clara si introducimos la formalización que van Inwagen propone: $Dx = _{df} \, \exists y (Sxy) \land \forall y ((Syx \land Nyx) \rightarrow y = x)$}

Como explicamos al inicio del texto, el incompatibilismo que van Inwagen presenta en \textit{An Essay on Free Will} supone el determinismo y de ahí prueba la incompatibilidad de este con el libre albedrío. Así, tras exponer los términos que utilizará en su argumento, van Inwagen pasa a afirmar que el determinismo puede ser representado satisfactoriamente como la expresión `$DA$', la cual afirma que el mundo actual es determinado. Esto no debe leerse como ‘nuestro mundo actual es determinado’, se trata más bien de definir el determinismo, no de afirmar que nuestro mundo es determinista, puede que sí, pero el punto de un incompatibilista como van Inwagen es que en todos los mundos posibles actualizados, comprensivos y determinados es imposible que exista el libre albedrío.

Para probarlo, van Inwagen proporciona una definición mínima de libre albedrío. Si logra probar que el determinismo es incompatible con esta definición, el determinismo es incompatible con cualquier otra versión del libre albedrío, pues esta nueva definición tendría que ser la versión mínima más algún otro elemento. La definición mínima es la siguiente: por lo menos un agente (en el pasado, presente o futuro) tuvo, tiene o tendrá acceso a un mundo posible distinto al mundo actual.

Así, van Inwagen afirma que la negación de la definición mínima de libre albedrío se sigue si decimos que el mundo actual es determinado (`$DA$') y otros dos principios metafísicos, los cuales llama `MAA' y `MAB':
\begin{itemize}
	\item MAA: Cualquier mundo al que una persona pueda acceder debe ser indistinguible del mundo actual por lo menos en algún instante.
	\item MAB: Ninguna persona tiene acceso a un mundo en el cual las leyes de la naturaleza sean distintas a las del mundo actual.
\end{itemize}

van Inwagen deriva su incompatibilismo de la siguiente manera:
Supóngase que que el mundo actual es determinado (DA), acto seguido, supóngase que alguien, Daniela, en efecto tiene acceso a un mundo posible, ese mundo posible compartirá una tajada con el mundo actual y será nomológicamente congruente con el mundo actual, por los principios MAA y MAB; si un mundo determinado comparte una tajada con cualquier otro mundo y es nomológicamente congruente con dicho mundo posible, ambos mundos posibles son el mismo, por la definición de mundo posible determinado; si el mundo actual es determinado, entonces el mundo al que tiene acceso Daniela es el mundo actual, pues comparte una tajada con el mundo actual y es nomológicamente congruente con este. De esta manera es posible generalizar y decir que para cualquier mundo posible al que un agente pueda acceder es el mundo actual. Así concluimos diciendo que si el mundo actual es determinado, cualquier mundo al que un agente pueda acceder es el mundo actual.

Esto no es otra cosa, dice van Inwagen, que decir que si el mundo actual es determinado, nadie tiene acceso a un mundo no-actualizado. Esto es, por supuesto, la negación de la definición mínima del libre albedrío, la cual decía que por lo menos una persona tuvo, tiene o tendrá acceso a un mundo posible distinto al mundo actual.

Si bien el argumento de van Inwagen es bastante bueno, hay claros contraejemplos a este. Para presentar un contraejemplo basta con probar que alguien tiene acceso a un un mundo posible no actualizado, si es no actualizado, es distinto al mundo actual, pues el mundo actual está, por definición, actualizado. Para nosotros, el secreto para encontrar tal posibilidad está en el acceso que tenemos a algunas posibilidades futuras. Dentro de las posibilidades no actualizadas están tanto mundos posibles pasados que nunca fueron el caso (por ejemplo, un mundo posible donde Julio César no hubiera cruzado el Rubicón) como mundos posibles futuros, estados de cosas del mundo actual que no han llegado a ser el caso ¿es el argumento de van Inwagen capaz de demostrar que no tenemos acceso a estas dos clases de posibilidades no actualizadas? 

Creemos que no: el argumento de van Inwagen es suficiente para probar que no podemos acceder a formas en las que el mundo pudo ser, pero nuestra capacidad de acceder a mundos posibles futuros prueba que el libre albedrío en sentido mínimo es compatible con el determinismo de van Inwagen. De esta manera, es posible afirmar que un mundo posible determinado y actualizado es compatible con el libre albedrío en sentido mínimo.

Cualquier posibilidad futura es, por definición, distinta del mundo actual, pues una posibilidad futura no está actualizada. Tomemos `$FA$' y `$A$' para referirnos respectivamente a un estado futuro del mundo actual y al mundo actual, tal como van Inwagen lo define. Dado que `$FA$' es simplemente `$A$' en un instante posterior, las leyes de la naturaleza de `$FA$' son las mismas que en `$A$', pues nadie puede falsear una ley de la naturaleza. Pero también podemos establecer lo mismo aplicando el principio MAB: cualquier mundo posible al que tenga acceso tendrá las mismas leyes de la naturaleza que el mundo actual

Así, podemos defender que `$FA$' es nomológicamente congruente con `$A$'. Pero además, podemos decir que `$FA$' y `$A$' comparten una tajada, pues al fin y al cabo `$FA$' es solo una versión futura de `$A$', debe haber por lo menos un instante en el que sean indistinguibles. Ahora bien, si esto es así, `$FA$' es distinto a `$A$', pero nomológicamente congruente con este y ambos comparten una tajada.

Si esto es así, de cualquier mundo actualizado posible se puede afirmar que la versión futura del mundo actualizado es nomológicamente congruente y comparte una tajada con el mundo actual, pero es distinto de este. Por lo tanto, para cualquier mundo actualizado que sea determinado, su versión futura es nomológicamente congruente, comparten una tajada pero son distintos.

Ahora bien, el incompatibilismo afirma que que en ningún mundo posible el determinismo es compatible con el libre albedrío. Pero bien, en algunos mundos posibles actualizados determinados es posible acceder a alguna versión futura de ese este: si alguien conoce las leyes de la naturaleza de ese mundo, podrá predecir el estado futuro del mundo con total precisión, esto es, acceder a un mundo posible determinado, nomológicamente congruente con el mundo actual y que comparte una tajada con este último. Por lo tanto, este agente podría acceder a un mundo posible distinto al mundo actual a pesar del determinismo del mundo en el que vive.

Así, es posible afirmar que lo mínimo que se necesita en un mundo posible para que sean compatibles el determinismo y el libro albedrío (en sentido mínimo) es las leyes de la naturaleza puedan ser conocidas por los agentes de ese mundo. Por supuesto, esto vale para las definiciones de determinismo y libre albedrío propuestas por van Inwagen. Sin embargo, de ser esto cierto, se sigue que en los términos de van Inwagen no es posible construir una posición incompatibilista, pues sus definiciones y principios metafísicos permiten la existencia de mundos posibles donde son compatibles su determinismo con su libre albedrío en sentido mínimo.

%Pero si el argumento tiene contrajemplos ¿por qué es un argumento válido? Parte del problema del argumento es explicar con una una lógica meramente extensional como la lógica de primer orden, nociones que necesitan ser abordadas con ayuda de lógica modal alética o lógica temporal. Por ejemplo, la noción de ‘compartir un tajada con’ es formalizada por van Inwagen de la siguiente manera: $$Sxy = _{df} \, \exists t (\textrm{the state that \textit{x} is in at \textit{t} = the state that \textit{y} is in at \textit{t}})$$

	\subsection{La necesidad del pasado: en contra del tercer argumento}
	
\section{El meta-problema del compatibilismo}

\section{Conclusiones}

\newpage
\printbibliography[title={Referencias}]

\end{document}